% preamble
\documentclass[aspectratio=169]{beamer}
\usetheme{Madrid}
\usecolortheme{orchid}
\usepackage{graphicx}
\usepackage{dirtytalk}

% begin
\begin{document}
% title page
\title{Algorithms}
\author[Premchander]{Kanishk Premchander}
\institute[ICR Portland 2024]{Institute for Computing in Research}
\date[08/02/2024]{August 2, 2024}

\frame{\titlepage}

% section 1 - algorithm intro
\section{What is an Algorithm?}

\begin{frame}{What is an Algorithm?} \pause
    \begin{figure}
        \begin{minipage}{0.5\textwidth}
            \centering
            \includegraphics[width=0.75\textwidth]{images/alg-flow.png}
        \end{minipage}
        \begin{minipage}{0.5\textwidth}
            \centering
            \includegraphics[width=0.75\textwidth]{images/alg-flow2.png}
        \end{minipage}
    \end{figure}
\end{frame}

\begin{frame}{What is an Algorithm?}
    \centering
    \includegraphics[width=0.7\textwidth]{images/ikea-manual.png}
\end{frame}

\begin{frame}{What is an Algorithm?} \pause
    \begin{figure}
        \begin{minipage}{0.5\textwidth}
            \centering
            \includegraphics[width=\textwidth]{images/basic-alg-1.png}
            \caption{Simple Addition Algorithm} \pause
        \end{minipage}
        \begin{minipage}{0.5\textwidth}
            \centering
            \includegraphics[width=\textwidth]{images/basic-alg-1-out.png}
            \caption{Result of adder(5, 10)}
        \end{minipage}
    \end{figure}
\end{frame}

\begin{frame}{What is an Algorithm?} \pause
    \begin{figure}
        \begin{minipage}{\textwidth}
            \begin{minipage}{0.5\textwidth}
                \centering
                \includegraphics[width=\textwidth]{images/basic-alg-2.png}
                \caption{More Complex Quadratic Algorithm} \pause
            \end{minipage}%
            \begin{minipage}{0.5\textwidth}
                \centering
                \includegraphics[width=\textwidth]{images/basic-alg-2-print.png}
                \caption{Print Statements} \pause
            \end{minipage}
        \end{minipage}
        \begin{minipage}{0.5\textwidth}
            \centering
            \includegraphics[width=\textwidth]{images/basic-alg-2-out.png}
            \caption{Results of Using quadratic Algorithm}
        \end{minipage}
    \end{figure}
\end{frame}

% section 2 - search algorithms
\section{Search Algorithms}

% section 2.1 - linear search
\subsection{Linear Search}

\begin{frame}{Linear Search} \pause
    \begin{figure}
        \begin{center}
            \includegraphics[width=0.5\textwidth,keepaspectratio]{images/ls-flow.png}
            \caption{Linear Search Flowchart}
        \end{center}
    \end{figure}
\end{frame}

\begin{frame}{Linear Search}
    \begin{figure}
        \includegraphics[width=0.5\textwidth,keepaspectratio]{images/linear-search-code.png}
        \caption{Example Code}
    \end{figure}
\end{frame}

\begin{frame}{Linear Search}
    \begin{figure}
        \begin{minipage}{0.5\textwidth}
            \begin{center}
                \includegraphics[width=0.75\textwidth,keepaspectratio]{images/ls-out-1.png}
                \caption{Output 1}
            \end{center}
        \end{minipage}%
        \begin{minipage}{0.5\textwidth}
            \begin{center}
                \includegraphics[width=0.75\textwidth,keepaspectratio]{images/ls-out-2.png}
                \caption{Output 2}
            \end{center}
        \end{minipage}
    \end{figure}
\end{frame}

% section 2.2 - linear search
\subsection{Binary Search}

\begin{frame}{Binary Search} \pause
    \begin{figure}
        \begin{center}
            \includegraphics[width=0.5\textwidth,keepaspectratio]{images/bs-flow.png}
            \caption{Binary Search Flowchart}
        \end{center}
    \end{figure}
\end{frame}

\begin{frame}{Binary Search}
    \begin{figure}
        \centering
        \includegraphics[height=0.75\textheight]{images/binary-search-code.png}
        \caption{Example Code}
    \end{figure}
\end{frame}

\begin{frame}{Binary Search}
    \begin{figure}
        \begin{minipage}{0.5\textwidth}
            \centering
            \includegraphics[width=0.5\textwidth]{images/bs-out-even-low.png}
            \caption{Even-Low Output}
            \includegraphics[width=0.5\textwidth]{images/bs-out-even-high.png}
            \caption{Even-High Output}
        \end{minipage}%
        \begin{minipage}{0.5\textwidth}
            \centering
            \includegraphics[width=0.5\textwidth]{images/bs-out-odd-low.png}
            \caption{Odd-Low Output}
            \includegraphics[width=0.5\textwidth]{images/bs-out-odd-high.png}
            \caption{Odd-High Output}
        \end{minipage}
    \end{figure}
\end{frame}

% section 3 - big O notation
\section{Big O Notation and Runtime}

\begin{frame}{Big O Notation and Runtime} \pause
    \[f(x)=3x^4+7x^3+x+1\] \pause

    Therefore, in Big O notation, the function would look something like this:
    \[f(x) \in O(x^4)\] \pause \newline
    \begin{center}
        $x^4$ time complexity
    \end{center}
\end{frame}

% section 4 - more on runtime
\section{Runtime Problems}

% section 4.1 - reasonable vs unreasonable time
\subsection{Reasonable vs Unreasonable Time}

\begin{frame}{Reasonable vs Unreasonable Time}
    Take a look at this sample table of runtime model orders and actual runtimes (in nanoseconds) for different input sizes:
    \begin{center}
        \begin{tabular}{ |c|c|c|c|c|c| } 
            \hline
            \mbox{} & $log(x)$ & $x$ & $x^2$ & $2^x$ & $x!$ \\
            \hline
            $2$ & $0.301$ & $2.00$ & $4.00$ & $4.00$ & $2.00$ \\
            $5$ & $0.699$ & $5.00$ & $2.50*10^1$ & $3.20*10^1$ & $1.20*10^2$ \\
            $10$ & $1.00$ & $10.0$ & $1.00*10^2$ & $1.02*10^3$ & $3.63*10^6$ \\
            $20$ & $1.30$ & $20.0$ & $4.00*10^2$ & $1.05*10^6$ & $2.43*10^{18}$ \\
            $50$ & $1.699$ & $50.0$ & $2.50*10^3$ & $1.13*10^{15}$ & $3.04*10^{64}$ \\
            \hline
        \end{tabular}
    \end{center}
\end{frame}

% section 4.2 - P vs NP
\subsection{The P vs NP Problem}

\begin{frame}{The P vs NP Problem} \pause
    \begin{itemize}
        \item \textbf{class-P} (\textit{polynomial} time) \pause
        \item \textbf{class-NP} (\textit{nondeterministic polynomial} time)
    \end{itemize}
\end{frame}

\begin{frame}{The P vs NP Problem} \pause
    \begin{figure}
        \begin{minipage}{0.5\textwidth}
            \begin{center}
                \includegraphics[width=\textwidth,keepaspectratio]{images/P<NP.png}
                \caption{P as a subset of NP} \pause
            \end{center}
        \end{minipage}%
        \begin{minipage}{0.5\textwidth}
            \begin{center}
                \includegraphics[width=\textwidth,keepaspectratio]{images/P=NP.png}
                \caption{P as equal to NP}
            \end{center}
        \end{minipage}
    \end{figure}
\end{frame}

\begin{frame}{A Quote For the Road} \pause
    \begin{center}
        \begin{minipage}{0.8\textwidth}
            \textit{Perhaps the most important principle for the good algorithm designer is to refuse to be content.}

            \hspace{1cm}---Alfred V. Aho
        \end{minipage}
    \end{center}
\end{frame}

% section 5 - bibliography
\section{Bibliography}

\begin{frame}{Bibliography}
    \begin{enumerate}
        \item Wikipedia. (n.d.). Big O Notation. Wikipedia. Retrieved July 29, 2024, from https://en.wikipedia.org/wiki/Big\_O\_notation
        \item Wikipedia. (n.d.). P versus NP problem. Wikipedia. Retrieved July 29, 2024, from https://en.wikipedia.org/wiki/P\_versus\_NP\_problem
    \end{enumerate}
\end{frame}

\end{document}